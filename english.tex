\documentclass[11pt,a4paper,sans]{moderncv}

% ModernCV setup
\moderncvstyle{classic}     % options: casual, classic, banking, oldstyle, fancy
\moderncvcolor{orange}      % base color (we will override with exact hex)

% adjust the page margins
\setlength{\hintscolumnwidth}{2cm}
%\setlength{\makecvheadnamewidth}{11cm}

% font loading
% for luatex and xetex, do not use inputenc and fontenc
% see https://tex.stackexchange.com/a/496643
\ifxetexorluatex
  \usepackage{fontspec}
  \usepackage{unicode-math}
  \defaultfontfeatures{Ligatures=TeX}
  \setmainfont{Latin Modern Roman}
  \setsansfont{Latin Modern Sans}
  \setmonofont{Latin Modern Mono}
  \setmathfont{Latin Modern Math} 
\else
  \usepackage[T1]{fontenc}
  \usepackage{lmodern}
\fi

% document language
\usepackage[english]{babel}

% Encoding and geometry
\usepackage[utf8]{inputenc}
\usepackage[T1]{fontenc}
\usepackage[scale=0.82]{geometry}
\usepackage{xcolor}
\usepackage{enumitem}
\usepackage[pdfpagelabels=false]{hyperref}
\setlist[itemize]{noitemsep, topsep=0pt}

% Make sure moderncv uses our color definitions
\makeatletter
\providecommand*{\@firstcolor}{color1}
\makeatother

% Adjust hint column width (dates)
\setlength{\hintscolumnwidth}{3.6cm}

% Personal data
\name{Halvor}{Ødegård Teigen}
\title{Software Engineer}
\phone[mobile]{+47 97 47 00 77}
\email{halvorot@gmail.com}
\homepage{halvorteigen.no}
\social[linkedin]{halvor-teigen}
\social[github]{halvorot}
\photo[64pt][0pt]{halvor_portrett.jpg}

\begin{document}
\makecvtitle

\section{Profile}
\cvitem{}{
Halvor Ødegård Teigen works as a backend developer with a parallel focus on DevOps. He has a solid academic background with a Master's degree in cybernetics and robotics from NTNU. Through his four years of professional experience, he has demonstrated proficiency in Java, Kotlin, Go, Spring Boot, Kafka, and React. He has also gained experience in various roles such as team lead, tech lead, and fullstack developer as a part of interdisciplinary teams with agile development methodologies.
}
\cvitem{}{
Halvor is committed to continuous growth and has invested time in several certifications and courses. He has completed courses in both Java, Kotlin and AWS, is ISTQB-certified in several test areas, and is Kafka-accredited through Confluent.
}
\cvitem{}{
As a person, Halvor is easy going and eager to learn with an "Everything can be learned"-mentality. He has shown the ability to take responsibility and ownership of tasks, and is devoted to delivering work he can be proud of in hindsight. He keeps up to date with the state of the art, both within and outside of his own professional field, and his wide range of experiences has given him the ability to cooperate with different personalities and contribute to holistic solutions from an interdisciplinary perspective.
}

\section{Key qualifications}
\cvitem{Backend}{Kotlin, Java, Go, Spring Boot, Kafka}
\cvitem{Cloud \& DevOps}{AWS, Kubernetes, Docker, Serverless, GitLab, GitHub}
\cvitem{Database}{PostgreSQL, Hibernate, Sqlc}
\cvitem{Test tools}{Kotest, JUnit, Mockito, Mockk, Postman/Bruno, Testcontainers}
\cvitem{Other}{Machine learning, Optimization}

\section{Highlighted courses and certifications}
\cvitem{}{\begin{itemize}
  \item Kotlin for Java Developers
  \item Java SE 12 Programming
  \item GitLab Certified CI/CD Associate
  \item Confluent Kafka Fundamentals Accreditation
  \item AWS Certified Solutions Architect - Associate
  \item ISTQB Certified Tester, Foundation Level
\end{itemize}}

\clearpage

\section{Work experience - overview}
\cvitem{}{\textit{See 'Work experience - details' below for more detailed info on each experience.}}

\cvitem{10.2024--present}{\textbf{Hafslund Vekst} \-- Software Engineer}
\cvitem{08.2021--10.2024}{\textbf{Sopra Steria} \-- Software Engineer, DevOps}
\cvitem{05.2020--08.2020}{\textbf{Geilo Entreprenør} \-- Construction worker}
\cvitem{06.2019--08.2019}{\textbf{Kongsberg Maritime} \-- Summer intern, developer}
\cvitem{01.2018--06.2019}{\textbf{Norwegian University of Science and Technology (NTNU)} \-- Teaching Assistant}
\cvitem{02.2017--12.2017}{\textbf{Lumenia AS} \-- Sales representative}
\cvitem{03.2017--09.2017}{\textbf{Retail House Norway AS} \-- Promoter}
\cvitem{07.2015--06.2016}{\textbf{The Norwegian Army} \-- Military service, HMKG}
\cvitem{06.2014--08.2018}{\textbf{Geilo Entreprenør} \-- Construction worker}
\cvitem{05.2015--05.2015}{\textbf{Liodden Camping} \-- Web Developer}
\cvitem{2014}{\textbf{Ødegård Teigen Hytteutleie AS} \-- Web developer}
\cvitem{2012--2013}{\textbf{Expert Geilo} \-- Store employee}

\section{Projects - overview}
\cvitem{}{\textit{See 'Projects - details' below for more detailed info on each project.}}

\cvitem{05.2025--present}{\textbf{Hafslund Kraftmarked - KundeAPI} \-- Backend developer}
\cvitem{10.2024--present}{\textbf{Hafslund Fleksibilitetstjenester} \-- Backend developer, Interim Tech Lead}
\cvitem{05.2024--10.2024}{\textbf{BarentsWatch Lukkede Tjenester} \-- Fullstack developer}
\cvitem{08.2022--05.2024}{\textbf{Statnett - DevOps and Test automation} \-- Developer, Team lead}
\cvitem{03.2022--08.2022}{\textbf{Sopra Steria internal - Test Data Service} \-- Researcher, Developer}
\cvitem{09.2021--03.2022}{\textbf{Statens Vegvesen - Syntopia} \-- Data Scientist, Backend developer}
\cvitem{01.2021--06.2021}{\textbf{NTNU - Master Thesis} \-- Author, Developer}
\cvitem{08.2020--01.2021}{\textbf{NTNU - Specialization Project} \-- Author, Developer}
\cvitem{06.2019--08.2019}{\textbf{Kongsberg Maritime - SmartShip} \-- Developer}
\cvitem{01.2019--05.2019}{\textbf{NTNU - Project, Real Time Elevators} \-- Developer}

\section{Education}
\cventry{08.2016--06.2021}{Master of Technology, Cybernetics and Robotics}{NTNU - Norwegian University of Science and Technology}{}{}{
Halvor has a master's degree in cybernetics and robotics from NTNU with specialization in autonomous systems. Halvor's main interest has been to combine machine learning and control theory, and he wrote his master thesis about the use of machine learning in off-shore wind turbines.
}

\cventry{09.2019--03.2020}{Exchange Abroad}{University of California, Santa Barbara (UCSB)}{}{}{
Halvor spent his fourth year abroad at University of California, Santa Barbara. He took courses in machine learning, computer vision, nonlinear control theory, as well as subjects in other non-engineering fields.
}

\section{Volunteer work}
\cvitem{}{Volunteer UKA-17: Halvor volunteered during the UKA festival in 2017 where he worked as a bartender at Samfundet in Trondheim. (2017)}

\clearpage

\section{Courses}
\cvitem{}{%
\begin{itemize}
  \item Kotlin for Java Developers \hfill 05.2023
  \item Cyber Security Academy Foundation \hfill 11.2021
  \item Presentation Techniques - Foundation \hfill 10.2021
  \item Digital Presentation Techniques - Foundation \hfill 08.2021
  \item Architecting on AWS - Amazon Web Services \hfill 08.2021
  \item Java SE 12 Programming \hfill 08.2021
  \item Learning assistant training (LAOS) \hfill 04.2019
  \item 77-853: MOS: Microsoft Office OneNote 2010 \hfill 05.2014
\end{itemize}
}

\section{Certifications}
\cvitem{}{%
\begin{itemize}
  \item Professional Scrum with Kanban (PSK I) \hfill 01.2024
  \item GitLab Certified CI/CD Associate \hfill 08.2023
  \item Professional Scrum Developer I (PSD I) \hfill 05.2023
  \item Professional Scrum Master I (PSM I) \hfill 01.2023
  \item DP-100 Azure Data Scientist Associate \hfill 10.2022
  \item AZ-204: Azure Developer Associate \hfill 09.2022
  \item PRINCE2 Agile Foundation \hfill 07.2022
  \item Certified Tester AI Testing (CT-AI) \hfill 05.2022
  \item Confluent Kafka Fundamentals Accreditation \hfill 04.2022
  \item DP-900: Microsoft Azure Data Fundamentals \hfill 02.2022
  \item AZ-900: Microsoft Azure Fundamentals \hfill 02.2022
  \item AZ-900: Microsoft Azure Fundamentals \hfill 01.2022
  \item IREB Foundation Level Certified Professional for Requirements Engineering \hfill 10.2021
  \item ISTQB Foundation Level Certification Agile Tester \hfill 10.2021
  \item AWS Certified Solutions Architect - Associate \hfill 09.2021
  \item ISTQB Certified Tester, Foundation Level \hfill 08.2021
  \item 77-853: MOS: Microsoft Office OneNote 2010 \hfill 05.2014
\end{itemize}
}

\section{Publications}
\cvitem{09.2021}{\textit{Comparing deep reinforcement learning algorithms' ability to safely navigate challenging waters}, Frontiers in Robotics and AI.}

\section{Presentations and courses}
\cvitem{}{%
\begin{itemize}
  \item Techday 2023: Testautomation and testdata at Statnett \hfill 04.2023
  \item RUBIKS 2022: Differentially private synthetic data using deep learning \hfill 06.2022
  \item Lightning Talk at Testdagen ODIN: How to generate representative synthetic test data at Statens Vegvesen \hfill 11.2021
  \item Teaching Microsoft OneNote to high school students \hfill 05.2014
\end{itemize}
}

\section{Languages}
\cvitem{Norwegian}{Native speaker}
\cvitem{English}{Fluent}

\clearpage

\section{Work experience - details}\label{sec:work-experience-details}

\cvitem{\textbf{Employer:}}{\textbf{Hafslund Vekst}}
\cvitem{\textbf{Position:}}{Software Engineer}
\cvitem{\textbf{Duration:}}{10.2024--present}
\cvitem{}{Halvor is employed as a Software Engineer in the Fleksibilitetstjenester team in Hafslund Vekst AS.}

\cvitem{\textbf{Employer:}}{\textbf{Sopra Steria}}
\cvitem{\textbf{Position:}}{Software Engineer, DevOps}
\cvitem{\textbf{Duration:}}{08.2021--10.2024}
\cvitem{}{Halvor was employed as a Software Engineer in the software development division of Sopra Steria, specifically in the DevOps department.}

\cvitem{\textbf{Employer:}}{\textbf{Geilo Entreprenør}}
\cvitem{\textbf{Position:}}{Construction worker}
\cvitem{\textbf{Duration:}}{05.2020--08.2020}
\cvitem{}{The job consisted of construction work such as driving construction equipment, laying power and water lines, as well as developing plots.}

\cvitem{\textbf{Employer:}}{\textbf{Kongsberg Maritime}}
\cvitem{\textbf{Position:}}{Summer intern, developer}
\cvitem{\textbf{Duration:}}{06.2019--08.2019}
\cvitem{}{Halvor worked on the summer project SmartShip in the "Autonomous Control" group. Together with a team of about 12 students, he developed an autonomous ship. A scaled-down model of the Yara Birkeland ship was equipped with algorithms for autonomous waypoint following and obstacle avoidance. Halvor's main responsibility during the project was implementing object detection using radar imaging.}

\cvitem{\textbf{Employer:}}{\textbf{Norwegian University of Science and Technology (NTNU)}}
\cvitem{\textbf{Position:}}{Teaching Assistant}
\cvitem{\textbf{Duration:}}{01.2018--06.2019}
\cvitem{}{Halvor worked as a teaching assistant in Procedural and Object-Oriented Programming (C++) for two semesters (spring 2018 and spring 2019). He guided students attending the course and approved their assignments.}

\cvitem{\textbf{Employer:}}{\textbf{Lumenia AS}}
\cvitem{\textbf{Position:}}{Sales representative}
\cvitem{\textbf{Duration:}}{02.2017--12.2017}
\cvitem{}{Halvor worked with selling promotion to companies through advertisements in compendiums at NTNU.}

\cvitem{\textbf{Employer:}}{\textbf{Retail House Norway AS}}
\cvitem{\textbf{Position:}}{Promoter}
\cvitem{\textbf{Duration:}}{03.2017--09.2017}
\cvitem{}{Halvor worked as a promoter for various products for Retail House Norway.}

\cvitem{\textbf{Employer:}}{\textbf{The Norwegian Army}}
\cvitem{\textbf{Position:}}{Military service, HMKG}
\cvitem{\textbf{Duration:}}{07.2015--06.2016}
\cvitem{}{Halvor did his military service in His Majesty The King's Guard the year after high school.}

\cvitem{\textbf{Employer:}}{\textbf{Geilo Entreprenør}}
\cvitem{\textbf{Position:}}{Construction worker}
\cvitem{\textbf{Duration:}}{06.2014--08.2018}
\cvitem{}{The job consisted of construction work such as driving construction equipment, laying power and water lines, as well as developing plots.}

\cvitem{\textbf{Employer:}}{\textbf{Liodden Camping}}
\cvitem{\textbf{Position:}}{Web Developer}
\cvitem{\textbf{Duration:}}{05.2015--05.2015}
\cvitem{}{Developed the website for Liodden Camping using the software Adobe Muse.}

\cvitem{\textbf{Employer:}}{\textbf{Ødegård Teigen Hytteutleie AS}}
\cvitem{\textbf{Position:}}{Web developer}
\cvitem{\textbf{Duration:}}{2014}
\cvitem{}{Development of website (not current site) using Adobe Dreamweaver. Mainly with html and css.}

\cvitem{\textbf{Employer:}}{\textbf{Expert Geilo}}
\cvitem{\textbf{Position:}}{Store employee}
\cvitem{\textbf{Duration:}}{2012--2013}
\cvitem{}{Summer job in an electronics store the summer of 2012 and 2013.}

\section{Projects - details}\label{sec:project-details}

% ---- Hafslund Kraftmarked
\subsection{\textbf{Hafslund Kraft}}
\cvitem{\textbf{Assignment:}}{Hafslund Kraftmarked - KundeAPI}
\cvitem{\textbf{Competencies:}}{Go, PostgreSQL}
\cvitem{\textbf{Roles}}{%
\textbf{Backend developer:} Halvor is a backend developer on the KundeAPI initiative within Hafslund Kraftmarked. KundeAPI aims to create a unified API interface for customers to exchange data and trade volumes through Hafslund Kraft.
}

\cvitem{}{\rule{\linewidth}{\arrayrulewidth}}

% ---- Hafslund Fleksibilitetstjenester
\subsection{\textbf{Hafslund Vekst}}
\cvitem{\textbf{Assignment:}}{Hafslund Fleksibilitetstjenester}
\cvitem{\textbf{Competencies:}}{Kotlin, AWS, Timescale}
\cvitem{\textbf{Roles}}{%
\textbf{Backend developer:} Halvor is a backend developer for the new business initiative Hafslund Fleksibilitetstjenester, which operates as a start-up within Hafslund. The team consists of five people. The product focuses on a holistic approach to smart control of electrical loads and producers. This is done through spot price arbitrage, peak-shaving, and trading in both Statnett and local flexibility markets.
}

\cvitem{}{\rule{\linewidth}{\arrayrulewidth}}

% ---- BarentsWatch
\subsection{\textbf{BarentsWatch}}
\cvitem{\textbf{Assignment:}}{Lukkede Tjenester}
\cvitem{\textbf{Competencies:}}{Java, React.js, OpenAPI, TypeScript}
\cvitem{\textbf{Roles}}{%
\textbf{Fullstack developer:} Halvor worked across the entire stack with development in Java for the backend and React+TypeScript for the frontend. He was responsible for the implementation of new functionality from start to finish, as well as error corrections and improvements in the solution. He worked closely with the users, thus also had customer-related contact, requirements gathering and training as areas of responsibility.
}

\cvitem{}{\rule{\linewidth}{\arrayrulewidth}}

% ---- Statnett
\subsection{\textbf{Statnett}}
\cvitem{\textbf{Assignment:}}{TOD - Test automation and data}
\cvitem{\textbf{Competencies:}}{Apache Kafka, Test automation, Test data, Java, Spring Boot, REST API, Event-driven architecture, Docker, Red Hat OpenShift, GitLab CI/CD, Amazon S3, Customer contact, Team leadership, Requirements analysis, Kubernetes}
\cvitem{\textbf{Roles}}{%
\textbf{Developer:} Halvor followed the project throughout its entire lifecycle from inception to the maintenance phase. It started with requirement analysis and system architecture design to ensure a solution that met the needs of the customer teams.
He then worked with development in Java, both with and without the Spring boot framework. This consisted of the development of several applications, as well as the implementation of a Java framework that was adopted by the development teams in Statnett. Technologies that were used included Kafka, REST API, MSSQL database, and Amazon S3.
Halvor also had the main responsibility for the development of GitLab Pipelines and Ansible deploy for automated integration tests.
}
\cvitem{}{
\textbf{Team lead:} In march of 2023, Halvor took over the role as team lead while still being a developer in the team. As a team lead, Halvor had the overall responsibility for the team and their deliveries. He was responsible for planning and following up the deliveries and continuously ensuring that they were in line with the needs of the customer. This also involved securing financing and coordinating several customer teams which Team TOD delivered to. As part of an agile delivery team, Halvor held the role of Scrum Master during the common Scrum ceremonies.
}

\cvitem{}{\rule{\linewidth}{\arrayrulewidth}}

% ---- Sopra Steria
\subsection{\textbf{Sopra Steria - internal}}
\cvitem{\textbf{Assignment:}}{Service development - Test Data}
\cvitem{\textbf{Competencies:}}{GDPR, Service development, Data modeling, FastAPI, Sales, SQLAlchemy, React.js, Python, Pandas, Artificial intelligence (AI), Presentation skills, Differential Privacy}
\cvitem{\textbf{Roles}}{%
\textbf{Service developer:} Halvor established a strategy and a roadmap for further test data investment in Sopra Steria. This involved assessing and mapping different approaches to test data, identifying solution options, establishing and commercializing various service packages, and contributing with internal and external visibility.
Halvor has been the driving force behind the establishment of three service packages: Discover Test Data Privacy Insight, Accelerate Data Generation and Accelerate Test Data Management. He has also been on a number of customer visits, and lectured at both internal and external conferences - including Testdagen ODIN.
}
\cvitem{}{
Halvor's work has revitalized the test data area in Sopra Steria, and ensured that the field has clear and relevant service initiatives within analysis, masking, generation and management of test data. His work has been crucial in ensuring that Sopra Steria has a strong value message within the test data area in the market, which complies with the privacy requirements, and is in line with the GDPR legislation.
}
\cvitem{}{
\textbf{Researcher:} In connection with the development of a Proof-of-Concept and the definition of service packages, Halvor explored various approaches and technologies to solve the challenges associated with a solution for data generation. This included, among other things, surveys of frameworks, providers and technologies that could be relevant, as well as discussions and negotiations with relevant subcontractors.
}
\cvitem{}{
\textbf{Developer:} Halvor developed a Proof-of-Concept in Python for one of the services in question, Accelerate Data Generation. Functionality was developed for interaction with databases through SQLAlchemy, analysis and generation of synthetic data with SDV, built an API with FastAPI and a simple frontend in React for visualization of generated data from the API.
}

\cvitem{}{\rule{\linewidth}{\arrayrulewidth}}

% ---- Statens Vegvesen
\subsection{\textbf{Statens Vegvesen}}
\cvitem{\textbf{Assignment:}}{Syntopia}
\cvitem{\textbf{Competencies:}}{SQL, Python, Pandas, Atlassian Bitbucket, Java, Spring Boot, Pytest, JPA, Hibernate, Jenkins, GDPR, Postman, Test data}
\cvitem{Roles}{%
\textbf{Developer:} Halvor worked in an agile development team where he developed and designed a data generation tool that would create large amounts of synthetic data to effectively test a large IKT portfolio in Statens vegvesen. Halvor performed data flow analyzes to investigate the complex connection between the various IKT systems, which clarified the needs and solution space for the test data solution. Synthetic data were generated based on analyzes of current data and statistical models, so that the data was in accordance with GDPR legislation. Halvor also worked with the integration of the synthetic data up against the national synthetic population register - Tenor.
}
\cvitem{}{
The tool was written in Python, and SQL was used to retrieve and insert data into databases. Communication with the Oracle databases was done using the cx\_Oracle library and Pandas and NumPy were used for data analysis. Jira and Bitbucket were used to integrate DevOps workflows. The test library Pytest was used to test that the generated data maintained its integrity with respect to references and limitations given by the database schema.
}
\cvitem{}{
Halvor planned and conducted workshops with testers and developers within Statens Vegvesen to map their needs and requirements for the solution. He also conducted workshops with external actors with the purpose of gaining experience from other similar projects.
}

\cvitem{}{%
\textbf{Backend developer:} Halvor contributed to the development of a plugin-based system (Syntopia) for the administration of test data across the IKT systems in Statens vegvesen. Halvor implemented search and deletion of data in databases based on given criteria through the existing plugin framework. The project was developed in Java using technologies like the Spring framework, hibernate, JPA, JSON schema, and was deployed via OpenShift. Confluence and Jira were used for documentation and work tracking.
}

\cvitem{}{\rule{\linewidth}{\arrayrulewidth}}

% ---- NTNU Master Thesis
\subsection{\textbf{Norwegian University of Science and Technology (NTNU)}}
\cvitem{\textbf{Assignment:}}{Master Thesis}
\cvitem{\textbf{Competencies:}}{Python, Machine learning, Git, Simulation, Artificial intelligence (AI), LaTeX, Matplotlib, NumPy, Mathematical modeling
}
\cvitem{\textbf{Roles}}{%
\textbf{Project description:} Halvor's master thesis focuses on stabilization of off-shore windturbines using machine learning, specifically Safe Reinforcement Learning. Off-shore windpower is becoming increasingly important on the path towards a more sustainable world. Due to high wind speeds and large waves, the turbines experience extremely large destabilizing forces, thus active stabilization can increase both safety and efficiency.
}
\cvitem{}{
Traditional control methods require mathematical models of the turbine dynamics. These are known to be complicated and hard to derive control laws for. This master thesis takes an alternative approach by using Reinforcement Learning, where an agent teaches itself the dynamics and the optimal action in a given state. The lack of guarantees on constraint satisfaction is a root problem in machine learning, thus the thesis combined Reinforcement Learning with Predictive Safety Filtering to ensure constraint satisfaction.
}
\cvitem{}{
The thesis was written as a joint master thesis with one other student. The thesis was also nominated for Norwegian Open AI Lab's "Best AI Master's Thesis Award 2021".
}
\cvitem{}{
\textbf{Developer \& Author:} The implementation is in Python and the OpenAI Gym framework. An extensive report is also written, which includes both results and the required theoretical background.
}

\cvitem{}{\rule{\linewidth}{\arrayrulewidth}}

% ---- NTNU Specialization Project
\subsection{\textbf{Norwegian University of Science and Technology (NTNU)}}
\cvitem{\textbf{Assignment:}}{Specialization Project}
\cvitem{\textbf{Competencies:}}{Deep Learning, Machine learning, Artificial intelligence (AI), LaTeX, Python
}
\cvitem{\textbf{Roles}}{%
\textbf{Project description:} Halvor's work on the specialization project explored various Deep Reinforcement Learning algorithms and investigated their performance in the application of path-following and obstacle-avoidance for autonomous vessels. This was done through training of multiple agents for each algorithm as well as extensive generalization testing. A custom performance function was developed to create a quantitative measure of performance for comparison of the selected algorithms. The project also led to a publication in the journal Frontiers in Robotics and AI.
}
\cvitem{}{
\textbf{Developer \& Author:} The implementation is in Python and the OpenAI Gym framework. An extensive report is also written, which includes both results and the required theoretical background.
}

\cvitem{}{\rule{\linewidth}{\arrayrulewidth}}

% ---- Kongsberg Maritime
\subsection{\textbf{Kongsberg Maritime}}
\cvitem{\textbf{Assignment:}}{SmartShip}
\cvitem{\textbf{Competencies:}}{C++, Scrum, Azure DevOps, Multidisciplinary team, OpenCV, Robotics, Artificial intelligence (AI), Radar technology
}
\cvitem{\textbf{Roles}}{%
\textbf{Project description:} Halvor worked on the summer project SmartShip in the "Autonomous Control" group. Together with a team of about 12 students, he developed an autonomous ship. This was a colaboration between three groups: Autonomous control, Cyber security and Shore control center. A scaled-down model of the Yara Birkeland ship was equipped with algorithms for autonomous waypoint following and obstacle avoidance. Halvor implemented the object detection using radar imaging for use in the obstacle avoidance algorithm.
}
\cvitem{}{
On this project, Halvor worked mainly with technologies such as the ROS framework, C++, OpenCV, Azure boards and repos, as well as position and velocity estimation using Kalman filters.
}
\cvitem{}{
\textbf{Developer:} Development of software and algorithms for autonomous vessels, as well as real world testing by regularly lauching a scaled-down ship model. Programming mainly in C++ and using ROS. Halvor both worked on a team of 5 people, and cooperated with the other groups on the project.
}

\cvitem{}{\rule{\linewidth}{\arrayrulewidth}}

% ---- NTNU Project - Real Time Elevators
\subsection{\textbf{Norwegian University of Science and Technology (NTNU)}}
\cvitem{\textbf{Assignment:}}{Project - Real Time Elevators}
\cvitem{\textbf{Competencies:}}{Golang, User Datagram Protocol (UDP), Real-time systems, Fault tolerance
}
\cvitem{\textbf{Project description:}}{%
As part of the subject TTK4145 - Real-time programming, Halvor carried out an elevator project where the goal was to program an arbitrary number of elevators to cooperate optimally over a given number of floors. A central part of the project was also to make the system fault-tolerant so that unforeseen events such as loss of power or network connection for one of the elevators did not affect functionality. Physical models of elevators with associated control panels were used. The lifts communicated over the network so that several units could work together. Logic for handling orders was implemented with the programming language Go and for the transfer of information over the network the protocol UDP was used.
}

\end{document}
