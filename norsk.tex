\documentclass[11pt,a4paper,sans]{moderncv}

% ModernCV setup
\moderncvstyle{classic}     % options: casual, classic, banking, oldstyle, fancy
\moderncvcolor{orange}      % base color (we will override with exact hex)

% font loading
% for luatex and xetex, do not use inputenc and fontenc
% see https://tex.stackexchange.com/a/496643
\ifxetexorluatex
  \usepackage{fontspec}
  \usepackage{unicode-math}
  \defaultfontfeatures{Ligatures=TeX}
  \setmainfont{Latin Modern Roman}
  \setsansfont{Latin Modern Sans}
  \setmonofont{Latin Modern Mono}
  \setmathfont{Latin Modern Math} 
\else
  \usepackage[T1]{fontenc}
  \usepackage{lmodern}
\fi

% document language
\usepackage[norwegian]{babel}

% Encoding and geometry
\usepackage[utf8]{inputenc}
\usepackage[scale=0.82]{geometry}
\usepackage{xcolor}
\usepackage{enumitem}
\usepackage[pdfpagelabels=false]{hyperref}
\setlist[itemize]{noitemsep, topsep=0pt}

% adjust the page margins
\setlength{\hintscolumnwidth}{3cm} % Adjust hint column width (dates)
%\setlength{\makecvheadnamewidth}{11cm}

% Personal data
\name{Halvor}{Ødegård Teigen}
\title{Software Engineer}
\homepage{halvorteigen.no}
\social[linkedin]{halvor-teigen}
\social[github]{halvorot}
\photo[64pt][0pt]{halvor_portrett.jpg}

\begin{document}
\makecvtitle

\section{Profil}
\cvitem{}{%
Halvor Ødegård Teigen jobber som backend-utvikler med parallelt fokus på DevOps. Han har en solid akademisk bakgrunn med en mastergrad i kybernetikk og robotikk fra NTNU. Gjennom sine fire år med profesjonell erfaring har han vist kompetanse i Java, Kotlin, Go, Spring Boot, Kafka og React. Han har også opparbeidet erfaring i roller som team lead, tech lead og fullstack-utvikler som del av tverrfaglige team med smidige arbeidsmetodikker.
}
\cvitem{}{%
Halvor er opptatt av å lære og utvikle seg, og har investert tid i en rekke sertifiseringer og kurs. Han har fullført kurs i både Java, Kotlin og AWS, er ISTQB-sertifisert innen flere testområder og er Kafka-akkreditert gjennom Confluent.
}
\cvitem{}{%
Som person er Halvor omgjengelig og svært lærevillig med en «Alt kan læres»-mentalitet. Han har vist evne til å ta ansvar og eierskap til oppgaver og er opptatt av å levere arbeid han kan være stolt av i ettertid. Han holder seg oppdatert på state-of-the-art både innenfor og utenfor eget fagfelt, og hans brede erfaring har gitt ham evnen til å samarbeide med ulike personligheter og bidra til helhetlige løsninger fra et tverrfaglig perspektiv.
}

\section{Nøkkelkvalifikasjoner}
\cvitem{Backend}{Kotlin, Java, Go, Spring Boot, Kafka}
\cvitem{Cloud \& DevOps}{AWS, Kubernetes, Docker, Serverless, GitLab, GitHub}
\cvitem{Database}{PostgreSQL, Hibernate, Sqlc}
\cvitem{Testverktøy}{Kotest, JUnit, Mockito, Mockk, Postman/Bruno, Testcontainers}
\cvitem{Annet}{Maskinlæring, Optimalisering}

\section{Utvalgte kurs og sertifiseringer}
\cvitem{}{\begin{itemize}
  \item Kotlin for Java Developers
  \item Java SE 12 Programming
  \item GitLab Certified CI/CD Associate
  \item Confluent Kafka Fundamentals Accreditation
  \item AWS Certified Solutions Architect - Associate
  \item ISTQB Certified Tester, Foundation Level
\end{itemize}}

\clearpage

\section{Arbeidserfaring - oversikt}
\cvitem{}{\textit{Se 'Arbeidserfaring - detaljer' nedenfor for mer detaljert informasjon om hver erfaring.}}

\cvitem{10.2024--nå}{\textbf{Hafslund Vekst} \-- Software Engineer}
\cvitem{08.2021--10.2024}{\textbf{Sopra Steria} \-- Software Engineer, DevOps}
\cvitem{05.2020--08.2020}{\textbf{Geilo Entreprenør} \-- Anleggsarbeider}
\cvitem{06.2019--08.2019}{\textbf{Kongsberg Maritime} \-- Sommerintern, utvikler}
\cvitem{01.2018--06.2019}{\textbf{NTNU} \-- Studentassistent}
\cvitem{02.2017--12.2017}{\textbf{Lumenia AS} \-- Salgsrepresentant}
\cvitem{03.2017--09.2017}{\textbf{Retail House Norway AS} \-- Promoter}
\cvitem{07.2015--06.2016}{\textbf{Forsvaret} \-- Verneplikt, HMKG}
\cvitem{06.2014--08.2018}{\textbf{Geilo Entreprenør} \-- Anleggsarbeider}
\cvitem{05.2015--05.2015}{\textbf{Liodden Camping} \-- Webutvikler}
\cvitem{2014}{\textbf{Ødegård Teigen Hytteutleie AS} \-- Webutvikler}
\cvitem{2012--2013}{\textbf{Expert Geilo} \-- Butikkmedarbeider}

\section{Prosjekter - oversikt}
\cvitem{}{\textit{Se 'Prosjekter - detaljer' nedenfor for mer detaljert informasjon om hvert prosjekt.}}

\cvitem{05.2025--nå}{\textbf{Hafslund Kraftmarked - KundeAPI} \-- Backend-utvikler}
\cvitem{10.2024--nå}{\textbf{Hafslund Fleksibilitetstjenester} \-- Backend-utvikler, Interim Tech Lead}
\cvitem{05.2024--10.2024}{\textbf{BarentsWatch Lukkede Tjenester} \-- Fullstack-utvikler}
\cvitem{08.2022--05.2024}{\textbf{Statnett - DevOps og Testautomatisering} \-- Utvikler, Team lead}
\cvitem{03.2022--08.2022}{\textbf{Sopra Steria intern - Test Data Service} \-- Researcher, Utvikler}
\cvitem{09.2021--03.2022}{\textbf{Statens Vegvesen - Syntopia} \-- Data Scientist, Backend-utvikler}
\cvitem{01.2021--06.2021}{\textbf{NTNU - Masteroppgave} \-- Forfatter, Utvikler}
\cvitem{08.2020--01.2021}{\textbf{NTNU - Fordypningsprosjekt} \-- Forfatter, Utvikler}
\cvitem{06.2019--08.2019}{\textbf{Kongsberg Maritime - SmartShip} \-- Utvikler}

\section{Utdanning}
\cventry{08.2016--06.2021}{Master i Kybernetikk og Robotikk}{NTNU - Norges teknisk-naturvitenskapelige universitet}{}{}{%
Halvor har en mastergrad i kybernetikk og robotikk fra NTNU med spesialisering i autonome systemer. Hans hovedinteresse har vært å kombinere maskinlæring og reguleringsteknikk, og han skrev masteroppgaven om bruk av maskinlæring i offshore vindmøller.
}

\cventry{09.2019--03.2020}{Utveksling}{University of California, Santa Barbara (UCSB)}{}{}{%
Halvor tilbrakte sitt fjerde år i utlandet ved University of California, Santa Barbara. Han tok fag innen maskinlæring, datamaskinsyn, ikke-lineær reguleringsteori samt emner i andre ikke-tekniske felt.
}

\section{Frivillig arbeid}
\cvitem{}{Frivillig UKA-17: Halvor var frivillig under UKA-festivalen i 2017 hvor han jobbet som bartender på Samfundet i Trondheim. (2017)}

\clearpage

\section{Kurs}
\cvitem{}{%
\begin{itemize}
  \item Kotlin for Java Developers \hfill 05.2023
  \item Cyber Security Academy Foundation \hfill 11.2021
  \item Presentation Techniques - Foundation \hfill 10.2021
  \item Digital Presentation Techniques - Foundation \hfill 08.2021
  \item Architecting on AWS - Amazon Web Services \hfill 08.2021
  \item Java SE 12 Programming \hfill 08.2021
  \item Learning assistant training (LAOS) \hfill 04.2019
  \item 77-853: MOS: Microsoft Office OneNote 2010 \hfill 05.2014
\end{itemize}
}

\section{Sertifiseringer}
\cvitem{}{%
\begin{itemize}
  \item Professional Scrum with Kanban (PSK I) \hfill 01.2024
  \item GitLab Certified CI/CD Associate \hfill 08.2023
  \item Professional Scrum Developer I (PSD I) \hfill 05.2023
  \item Professional Scrum Master I (PSM I) \hfill 01.2023
  \item DP-100 Azure Data Scientist Associate \hfill 10.2022
  \item AZ-204: Azure Developer Associate \hfill 09.2022
  \item PRINCE2 Agile Foundation \hfill 07.2022
  \item Certified Tester AI Testing (CT-AI) \hfill 05.2022
  \item Confluent Kafka Fundamentals Accreditation \hfill 04.2022
  \item DP-900: Microsoft Azure Data Fundamentals \hfill 02.2022
  \item AZ-900: Microsoft Azure Fundamentals \hfill 02.2022
  \item AZ-900: Microsoft Azure Fundamentals \hfill 01.2022
  \item IREB Foundation Level Certified Professional for Requirements Engineering \hfill 10.2021
  \item ISTQB Foundation Level Certification Agile Tester \hfill 10.2021
  \item AWS Certified Solutions Architect - Associate \hfill 09.2021
  \item ISTQB Certified Tester, Foundation Level \hfill 08.2021
  \item 77-853: MOS: Microsoft Office OneNote 2010 \hfill 05.2014
\end{itemize}
}

\section{Publikasjoner}
\cvitem{09.2021}{\textit{Comparing deep reinforcement learning algorithms' ability to safely navigate challenging waters}, Frontiers in Robotics and AI.}

\section{Presentasjoner og foredrag}
\cvitem{}{%
\begin{itemize}
  \item Techday 2023: Testautomatisering og testdata hos Statnett \hfill 04.2023
  \item RUBIKS 2022: Differensielt private syntetiske data med dyp læring \hfill 06.2022
  \item Lyntale på Testdagen ODIN: Hvordan generere representativ syntetisk testdata hos Statens Vegvesen \hfill 11.2021
  \item Undervisning i Microsoft OneNote for videregående elever \hfill 05.2014
\end{itemize}
}

\section{Språk}
\cvitem{}{\textbf{Norsk} - Morsmål}
\cvitem{}{\textbf{Engelsk} - Flytende}

\clearpage

\section{Arbeidserfaring - detaljer}\label{sec:work-experience-details}

\cvitem{\textbf{Arbeidsgiver:}}{\textbf{Hafslund Vekst}}
\cvitem{\textbf{Stilling:}}{Software Engineer}
\cvitem{\textbf{Varighet:}}{10.2024--nå}
\cvitem{}{Halvor er ansatt som Software Engineer i teamet Fleksibilitetstjenester i Hafslund Vekst AS.}

\cvitem{\textbf{Arbeidsgiver:}}{\textbf{Sopra Steria}}
\cvitem{\textbf{Stilling:}}{Software Engineer, DevOps}
\cvitem{\textbf{Varighet:}}{08.2021--10.2024}
\cvitem{}{Halvor var ansatt som Software Engineer i utviklingsdivisjonen i Sopra Steria, spesifikt i DevOps-avdelingen.}

\cvitem{\textbf{Arbeidsgiver:}}{\textbf{Geilo Entreprenør}}
\cvitem{\textbf{Stilling:}}{Anleggsarbeider}
\cvitem{\textbf{Varighet:}}{05.2020--08.2020}
\cvitem{}{Arbeidet bestod av anleggsarbeid som kjøring av maskiner, legging av strøm- og vannlinjer og utvikling av tomter.}

\cvitem{\textbf{Arbeidsgiver:}}{\textbf{Kongsberg Maritime}}
\cvitem{\textbf{Stilling:}}{Sommerstudent, utvikler}
\cvitem{\textbf{Varighet:}}{06.2019--08.2019}
\cvitem{}{Halvor jobbet på sommerprosjektet SmartShip i gruppen «Autonomous Control». Sammen med et team på rundt 12 studenter utviklet han et autonomt skip. En nedskalert modell av Yara Birkeland ble utstyrt med algoritmer for autonom ruteoppfølging og hindringsunngåelse. Halvors hovedansvar var implementering av objektdeteksjon ved bruk av radarbilder.}

\cvitem{\textbf{Arbeidsgiver:}}{\textbf{Norges teknisk-naturvitenskapelige universitet (NTNU)}}
\cvitem{\textbf{Stilling:}}{Studentassistent}
\cvitem{\textbf{Varighet:}}{01.2018--06.2019}
\cvitem{}{Halvor jobbet som studentassistent i Prosedyre- og Objektorientert Programmering (C++) i to semestre (vår 2018 og vår 2019). Han veiledet studenter og godkjente oppgaver.}

\cvitem{\textbf{Arbeidsgiver:}}{\textbf{Lumenia AS}}
\cvitem{\textbf{Stilling:}}{Salgsrepresentant}
\cvitem{\textbf{Varighet:}}{02.2017--12.2017}
\cvitem{}{Halvor jobbet med salg av promotering til bedrifter gjennom annonser i kompendier ved NTNU.}

\cvitem{\textbf{Arbeidsgiver:}}{\textbf{Retail House Norway AS}}
\cvitem{\textbf{Stilling:}}{Promotør}
\cvitem{\textbf{Varighet:}}{03.2017--09.2017}
\cvitem{}{Halvor jobbet som promotør for ulike produkter for Retail House Norway.}

\cvitem{\textbf{Arbeidsgiver:}}{\textbf{Forsvaret}}
\cvitem{\textbf{Stilling:}}{Verneplikt, HMKG}
\cvitem{\textbf{Varighet:}}{07.2015--06.2016}
\cvitem{}{Halvor gjennomførte førstegangstjenesten i Hans Majestet Kongens Garde året etter videregående.}

\cvitem{\textbf{Arbeidsgiver:}}{\textbf{Geilo Entreprenør}}
\cvitem{\textbf{Stilling:}}{Anleggsarbeider}
\cvitem{\textbf{Varighet:}}{06.2014--08.2018}
\cvitem{}{Arbeidet bestod av anleggsarbeid som kjøring av maskiner, legging av strøm- og vannlinjer og utvikling av tomter.}

\cvitem{\textbf{Arbeidsgiver:}}{\textbf{Liodden Camping}}
\cvitem{\textbf{Stilling:}}{Webutvikler}
\cvitem{\textbf{Varighet:}}{05.2015--05.2015}
\cvitem{}{Utviklet nettsiden for Liodden Camping ved bruk av Adobe Muse.}

\cvitem{\textbf{Arbeidsgiver:}}{\textbf{Ødegård Teigen Hytteutleie AS}}
\cvitem{\textbf{Stilling:}}{Webutvikler}
\cvitem{\textbf{Varighet:}}{2014}
\cvitem{}{Utvikling av nettside (ikke dagens) med Adobe Dreamweaver, hovedsakelig HTML og CSS.}

\cvitem{\textbf{Arbeidsgiver:}}{\textbf{Expert Geilo}}
\cvitem{\textbf{Stilling:}}{Butikkmedarbeider}
\cvitem{\textbf{Varighet:}}{2012--2013}
\cvitem{}{Sommerjobb i elektronikkbutikk sommeren 2012 og 2013.}

\section{Prosjekter - detaljer}\label{sec:project-details}

% ---- Hafslund Kraftmarked
\subsection{\textbf{Hafslund Kraft}}
\cvitem{\textbf{Oppdrag:}}{Hafslund Kraftmarked - KundeAPI}
\cvitem{\textbf{Kompetanser:}}{Go, PostgreSQL}
\cvitem{\textbf{Roller}}{%
\textbf{Backend-utvikler:} Halvor er backend-utvikler på KundeAPI-initiativet i Hafslund Kraftmarked. KundeAPI har som mål å etablere et samlet API-grensesnitt for kunder til å utveksle data og handle volum gjennom Hafslund Kraft.
}

\cvitem{}{\rule{\linewidth}{\arrayrulewidth}}

% ---- Hafslund Fleksibilitetstjenester
\subsection{\textbf{Hafslund Vekst}}
\cvitem{\textbf{Oppdrag:}}{Hafslund Fleksibilitetstjenester}
\cvitem{\textbf{Kompetanser:}}{Kotlin, AWS, Timescale}
\cvitem{\textbf{Roller}}{%
\textbf{Backend-utvikler:} Halvor er backend-utvikler for det nye forretningsinitiativet Hafslund Fleksibilitetstjenester, som opererer som en start-up i Hafslund. Teamet består av fem personer. Produktet fokuserer på en helhetlig tilnærming til smart styring av elektriske laster og produsenter gjennom spotpris-arbitrasje, peak-shaving og handel i både Statnett-markeder og lokale fleksibilitetsmarkeder.
}

\cvitem{}{\rule{\linewidth}{\arrayrulewidth}}

% ---- BarentsWatch
\subsection{\textbf{BarentsWatch}}
\cvitem{\textbf{Oppdrag:}}{Lukkede Tjenester}
\cvitem{\textbf{Kompetanser:}}{Java, React.js, OpenAPI, TypeScript}
\cvitem{\textbf{Roller}}{%
\textbf{Fullstack-utvikler:} Halvor jobbet på tvers av hele stacken med utvikling i Java for backend og React+TypeScript for frontend. Han hadde ansvar for implementering av ny funksjonalitet fra start til slutt, samt feilrettinger og forbedringer. Han jobbet tett med brukerne og hadde dermed også ansvarsområder som kundekontakt, behovskartlegging og opplæring.
}

\cvitem{}{\rule{\linewidth}{\arrayrulewidth}}

% ---- Statnett
\subsection{\textbf{Statnett}}
\cvitem{\textbf{Oppdrag:}}{TOD - Testautomatisering og -data)}
\cvitem{\textbf{Kompetanser:}}{Apache Kafka, Testautomatisering, Testdata, Java, Spring Boot, REST API, Event-drevet arkitektur, Docker, Red Hat OpenShift, GitLab CI/CD, Amazon S3, Kundekontakt, Teamledelse, Behovsanalyse, Kubernetes}
\cvitem{\textbf{Roller}}{%
\textbf{Utvikler:} Halvor fulgte prosjektet gjennom hele livsløpet fra oppstart til forvaltningsfase. Det startet med behovsanalyse og systemarkitektur for å sikre en løsning som møtte behovene til kundeteamene. Deretter jobbet han med utvikling i Java, både med og uten Spring Boot. Dette bestod av utvikling av flere applikasjoner, samt implementering av et Java-rammeverk som ble tatt i bruk av utviklingsteam i Statnett. Teknologier som ble brukt inkluderte Kafka, REST API, MSSQL og Amazon S3. Halvor hadde også hovedansvar for utvikling av GitLab Pipelines og Ansible deploy for automatiserte integrasjonstester.
}
\cvitem{}{%
\textbf{Team lead:} I mars 2023 tok Halvor over rollen som team lead samtidig som han fortsatt var utvikler i teamet. Som team lead hadde han det overordnede ansvaret for teamet og leveransene. Han var ansvarlig for planlegging og oppfølging av leveranser og for å sikre at de kontinuerlig var i tråd med kundens behov. Dette innebar også å sikre finansiering og koordinere flere kundeteam som Team TOD leverte til. Som del av et smidig leveringsteam hadde Halvor rollen som Scrum Master i felles Scrum-seremonier.
}

\cvitem{}{\rule{\linewidth}{\arrayrulewidth}}

% ---- Sopra Steria
\subsection{\textbf{Sopra Steria - intern}}
\cvitem{\textbf{Oppdrag:}}{Tjenesteutvikling - Test Data}
\cvitem{\textbf{Kompetanser:}}{GDPR, Tjenesteutvikling, Datamodellering, FastAPI, Salg, SQLAlchemy, React.js, Python, Pandas, Artificial intelligence (AI), Presentasjonsevner, Differential Privacy}
\cvitem{\textbf{Roller}}{%
\textbf{Tjenesteutvikler:} Halvor etablerte strategi og roadmap for videre satsning på testdata i Sopra Steria. Dette innebar vurdering og kartlegging av ulike tilnærminger til testdata, identifisering av løsningsalternativer, etablering og kommersialisering av ulike tjenestepakker, samt bidrag til intern og ekstern synlighet.
}
\cvitem{}{%
Halvor har vært drivkraften bak etableringen av tre tjenestepakker: Discover Test Data Privacy Insight, Accelerate Data Generation og Accelerate Test Data Management. Han har også vært på en rekke kundebesøk og holdt foredrag på både interne og eksterne konferanser – inkludert Testdagen ODIN.
}
\cvitem{}{%
Halvors arbeid har revitalisert testdataområdet i Sopra Steria og sikret at feltet har tydelige og relevante tjenesteinitiativ innen analyse, maskering, generering og forvaltning av testdata. Arbeidet har vært sentralt i å sikre at Sopra Steria har et sterkt verdibudskap i markedet som er i samsvar med personvernkrav og GDPR-lovgivning.
}
\cvitem{}{%
\textbf{Researcher:} I forbindelse med utviklingen av et Proof-of-Concept og definisjonen av tjenestepakker, utforsket Halvor ulike tilnærminger og teknologier for å løse utfordringene knyttet til datagenerering. Dette inkluderte kartlegging av rammeverk, leverandører og teknologier, samt diskusjoner og forhandlinger med relevante underleverandører.
}
\cvitem{}{%
\textbf{Utvikler:} Halvor utviklet et Proof-of-Concept i Python for en av tjenestene, Accelerate Data Generation. Denne inkluderte funksjonalitet for interaksjon med databaser gjennom SQLAlchemy, analyse og generering av syntetiske data med SDV, bygging av et API med FastAPI og en enkel frontend i React for visualisering av genererte data fra API-et.
}

\cvitem{}{\rule{\linewidth}{\arrayrulewidth}}

% ---- Statens Vegvesen
\subsection{\textbf{Statens Vegvesen}}
\cvitem{\textbf{Oppdrag:}}{Syntopia}
\cvitem{\textbf{Kompetanser:}}{SQL, Python, Pandas, Atlassian Bitbucket, Java, Spring Boot, Pytest, JPA, Hibernate, Jenkins, GDPR, Postman, Testdata}
\cvitem{Roles}{%
\textbf{Utvikler:} Halvor jobbet i et smidig utviklingsteam der han utviklet og designet et datagenereringsverktøy som skulle skape store mengder syntetiske data for effektiv testing av en stor IKT-portefølje i Statens vegvesen. Han gjorde dataflytanalyser for å undersøke komplekse sammenhenger mellom ulike IKT-systemer, noe som tydeliggjorde behov og løsningsrom for testdataverktøyet. Syntetiske data ble generert basert på analyser av eksisterende data og statistiske modeller, slik at dataene var i samsvar med GDPR. Halvor jobbet også med integrasjon av de syntetiske dataene mot det nasjonale syntetiske folkeregisteret - Tenor.
}
\cvitem{}{%
Halvor planla og gjennomførte workshops med testere og utviklere i Statens Vegvesen for å kartlegge deres behov og krav til løsningen. Han gjennomførte også workshops med eksterne aktører for å hente erfaring fra andre lignende prosjekter.
}

\cvitem{}{%
\textbf{Backend-utvikler:} Halvor bidro til utvikling av et plugin-basert system (Syntopia) for administrasjon av testdata på tvers av IKT-systemene i Statens vegvesen. Han implementerte søk, endring og sletting av data i databaser basert på gitte kriterier gjennom det eksisterende plugin-rammeverket. Prosjektet ble utviklet i Java med teknologier som Spring Boot, Hibernate, JPA, JSON schema og ble deployet via OpenShift. Confluence og Jira ble brukt til dokumentasjon og oppgavestyring.
}

\cvitem{}{\rule{\linewidth}{\arrayrulewidth}}

% ---- NTNU Masteroppgave
\subsection{\textbf{Norges teknisk-naturvitenskapelige universitet (NTNU)}}
\cvitem{\textbf{Oppdrag:}}{Masteroppgave - Reinforcement Learning and Predictive Safety Filtering for Floating Offshore Wind Turbine Control}
\cvitem{\textbf{Kompetanser:}}{Python, Maskinlæring, Git, Simulering, Artificial intelligence (AI), LaTeX, Matplotlib, NumPy, Matematisk modellering
}
\cvitem{\textbf{Roller}}{%
\textbf{Prosjektbeskrivelse:} Halvors masteroppgave fokuserer på stabilisering av offshore vindturbiner ved hjelp av maskinlæring, spesifikt Safe Reinforcement Learning. Offshore vindkraft blir stadig viktigere på veien mot en mer bærekraftig verden. På grunn av høye vindhastigheter og store bølger utsettes turbinene for store destabiliserende krefter; aktiv stabilisering kan derfor øke både sikkerhet og effektivitet.
}
\cvitem{}{%
Tradisjonelle reguleringsmetoder krever matematiske modeller av turbindynamikken. Disse er kompliserte og det er utfordrende å derivere reguleringslover. Oppgaven tar en alternativ tilnærming ved å bruke Reinforcement Learning, der en agent lærer seg dynamikken og optimal handling i en gitt tilstand. Mangelen på garantier for begrensningsoppfyllelse er et grunnproblem i maskinlæring, og oppgaven kombinerte derfor Reinforcement Learning med Predictive Safety Filtering for å sikre begrensningsoppfyllelse.
}
\cvitem{}{%
Oppgaven ble skrevet som en felles masteroppgave med én annen student. Oppgaven ble også nominert til Norwegian Open AI Lab sin "Best AI Master's Thesis Award 2021". Oppgaven finner du på NTNUs sider her: \url{https://ntnuopen.ntnu.no/ntnu-xmlui/handle/11250/2781089}
}
\cvitem{}{%
\textbf{Utvikler \& Forfatter:} Implementasjonen ble gjort i Python og benyttet OpenAI Gym-rammeverket. En omfattende rapport er også skrevet, som inkluderer både resultater og nødvendig teoretisk bakgrunn.
}

\cvitem{}{\rule{\linewidth}{\arrayrulewidth}}

% ---- NTNU Fordypningsprosjekt
\subsection{\textbf{Norges teknisk-naturvitenskapelige universitet (NTNU)}}
\cvitem{\textbf{Oppdrag:}}{Fordypningsprosjekt}
\cvitem{\textbf{Kompetanser:}}{Deep Learning, Maskinlæring, Artificial intelligence (AI), LaTeX, Python
}
\cvitem{\textbf{Roller}}{%
\textbf{Prosjektbeskrivelse:} Halvors arbeid i fordypningsprosjektet utforsket ulike Deep Reinforcement Learning-algoritmer og undersøkte deres ytelse i anvendelsen av ruteoppfølging og hindringsunngåelse for autonome fartøy. Dette ble gjort gjennom trening av flere agenter for hver algoritme samt omfattende generaliseringstesting. En egen ytelsesfunksjon ble utviklet for å skape et kvantitativt mål for sammenligning av utvalgte algoritmer. Prosjektet førte også til en publikasjon i tidsskriftet Frontiers in Robotics and AI.
}
\cvitem{}{%
\textbf{Utvikler \& Forfatter:} Implementasjonen er i Python og OpenAI Gym-rammeverket. En omfattende rapport er også skrevet, som inkluderer både resultater og nødvendig teoretisk bakgrunn.
}

\cvitem{}{\rule{\linewidth}{\arrayrulewidth}}

% ---- Kongsberg Maritime
\subsection{\textbf{Kongsberg Maritime}}
\cvitem{\textbf{Oppdrag:}}{SmartShip}
\cvitem{\textbf{Kompetanser:}}{C++, Scrum, Azure DevOps, Tverrfaglig team, OpenCV, Robotics, Artificial intelligence (AI), Radarteknologi
}
\cvitem{\textbf{Roller}}{%
\textbf{Prosjektbeskrivelse:} Halvor jobbet på sommerprosjektet SmartShip i gruppen «Autonomous Control». Sammen med et team på omtrent 12 studenter utviklet han et autonomt skip. Dette var et samarbeid mellom tre grupper: Autonomous Control, Cyber Security og Shore Control Center. En nedskalert modell av Yara Birkeland ble utstyrt med algoritmer for autonom ruteoppfølging og hindringsunngåelse. Halvor implementerte objektdeteksjon ved bruk av radarbilder for bruk i hindringsunngåelsesalgoritmen.
}
\cvitem{}{%
På dette prosjektet jobbet Halvor hovedsakelig med teknologier som ROS-rammeverket, C++, OpenCV, Azure Boards og Repos, samt posisjons- og hastighetsestimering ved bruk av Kalman-filtre.
}
\cvitem{}{%
\textbf{Utvikler:} Utvikling av programvare og algoritmer for autonome fartøy samt reell testing ved regelmessig utsettelse av en nedskalert modell. Programmering hovedsakelig i C++ og med ROS. Halvor jobbet både i et team på 5 personer og samarbeidet med de andre gruppene i prosjektet.
}

\end{document}
